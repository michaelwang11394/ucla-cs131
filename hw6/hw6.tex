% TEMPLATE for Usenix papers, specifically to meet requirements of
%  USENIX '05
% originally a template for producing IEEE-format articles using LaTeX.
%   written by Matthew Ward, CS Department, Worcester Polytechnic Institute.
% adapted by David Beazley for his excellent SWIG paper in Proceedings,
%   Tcl 96
% turned into a smartass generic template by De Clarke, with thanks to
%   both the above pioneers
% use at your own risk.  Complaints to /dev/null.
% make it two column with no page numbering, default is 10 point

% Munged by Fred Douglis <douglis@research.att.com> 10/97 to separate
% the .sty file from the LaTeX source template, so that people can
% more easily include the .sty file into an existing document.  Also
% changed to more closely follow the style guidelines as represented
% by the Word sample file. 

% Note that since 2010, USENIX does not require endnotes. If you want
% foot of page notes, don't include the endnotes package in the 
% usepackage command, below.

% This version uses the latex2e styles, not the very ancient 2.09 stuff.
\documentclass[letterpaper,twocolumn,10pt]{article}
\usepackage{usenix,epsfig,endnotes,url}
\begin{document}

%don't want date printed
\date{}

%make title bold and 14 pt font (Latex default is non-bold, 16 pt)
\title{\Large \bf CS 131 Homework 6: Containerization support languages}

%for single author (just remove % characters)
\author{
{\rm Zhehao Wang, 404380075}\\
zhehao@cs.ucla.edu
} % end author

\maketitle

% Use the following at camera-ready time to suppress page numbers.
% Comment it out when you first submit the paper for review.
\thispagestyle{empty}

\subsection*{Abstract}

In this report, we inspect alternatives languages to implement Docker, an open platform for building, shipping and running ``any'' applications ``anywhere''. The original Docker is implemented in Go, with features including utilizing Linux containers to build lightweight program execution environments, escaping the dependency hell problem, and easy to use, develop and maintain. This work studies the feasibility of using Java, Python or Scala to achieve the same functions as Docker provides, and analyzes the pros and cons of each choice. \cite{DockerSite}

\section{Docker in Go}

Before we dive into details of the inspected languages, it's important to understand what Docker hopes to provide, why it's significant, and why they chose Go language.

As the scale and complexity of software grows, software management and deployment have become more important. Commercial softwares today may face a variety of platforms, each with its own set of libraries providing certain functions to applications. For developers with the goal of distributing their software to different platforms, specific research usually need to be done for that platform, and often times different binaries will be needed for different platforms. To make matters worse, sometimes dependency issue's out of the control of the application developer: for example, when a dependency's not available or buggy for a platform.

The usual answer for this problem is a virtual machine. We can distribute a fully functioning virtual disk that carries the application as well as its dependencies, and the environment it's running under. The idea of Docker is similar, only that it's achieving the same purpose with a much more lightweight approach using Linux containers, an OS level virtualization environment for running multiple isolated Linux systems on a single host. A Linux container provides its own process space and network interface for Docker applications. \cite{LXC} \cite{DockerGithub}

Before choosing Go, the developers inspected other choices like Python, Ruby, Java, etc. Python is used in the developer's previous project, dotCloud, but the developers wanted to start from a clean slate with a more neutral, and less controversial language. An important concern here is adoption by ops: having a single binary that you can drop in any server is a big win, but Ruby shops don't Java, Python shops don't use Node, etc. Go is neutral in the sense that it's mostly platform agnostic, making it more suitable for this case. Go is designed with ``be statically typed'', ``be light on the page'' and ``support networking and multiprocessing natively'' in mind, and features \cite{GoSite} \cite{GoWikipedia}
\begin{itemize}
\item A C-like syntax, but more concise and readable
\item An expressive and lightweight type system that uses duck typing, and removes type hierarchy
\item A hybrid stop-the-world/concurrent mark-and-sweep garbage collector
\item A toolchain that by default generates statically linked native binaries without external dependencies
\item Built-in concurrency primitives: goroutines and channels
\item Remote package management system and hierarchical package naming
\item Fast compiler because of lightweight type analysis, and easy dependency analysis
\end{itemize}
Among these features, the developers of docker emphasized that a good package management system, a good standard library, and easy to read and contribute among their top reasons to choose Go. \cite{GopherCon2014}

\section{DockAlt in Java}

% Your summary should focus on the technologies' effects on ease of use, flexibility, generality, performance, reliability; thie idea is to explore the most-important technical challenges in doing the proposed rewrite

Java is one of the most widely used programming languages that features:
\begin{itemize}
\item Java is imperative and object oriented. \cite{JavaWikipedia}
\item Compiler compiles Java source code into byte code, which gets executed in a Java Virtual Machine.
\item Java type system is strongly and statically typed. It has type hierarchy, differentiates primitive and reference types, and is enhanced with explicit exception marking. \cite{JavaType}
\item Garbage collector is up to JVM implementation. Some commonly used JVMs use concurrent mark and sweep garbage collection.
\item Library has a large number of built-in classes that can handle various requirements.
\end{itemize}
The pros and cons of implementing DockAlt in Java include: 
\begin{itemize}
\item Pro: Implementation safety and reliability. In a large codebase like DockAlt, this is an important concern. Java type system helps in improving safety. By having a strongly typed language and compile time type checking, mistakes are more easily detected and reported at compile time. Java also has explicit exceptions in the type, which requires that exceptions thrown by the callee should be handled or passed on in the caller function, this is another feature that would improve code reliability. However, it's worth mentioning that because of the regulations this type system introduced, compilation tends to be slower. (Go compiler is fast, with one reason being that type relationships, such as the type hierarchy, is removed)
\item Pro: JVM provides some platform agnosticity. Java code can run on various platforms, mainly because of the existence of a corresponding JVM for that platform. This means for all the platforms that can run a standard JVM, only one version of DockAlt source is needed. Platform agnosticity is a plus for DockAlt, in this sense Java is similar with Go.
\item Con: Rather verbose code, longer development cycle and not very easy to use. Java may not be a good option for rapid prototyping purposes, or attracting community contributions. Docker as out right now has a large community, whose growth could be hindered if Java is chosen instead.
\item Con: Lacks LXC API. Although Java has a good number of useful built-in libraries, LXC interface in Java come as third party libraries, or bindings wrapping around an implementation in a different language. This will slow down the development cycle, add to the list of things that could go wrong, and add another factor that's potentially out of the control of the DockAlt developers.
\end{itemize}

\section{DockAlt in Python}

Python started out as a scripting language, and is often used for fast prototyping. \cite{PythonWikipedia}The features of Python include:
\begin{itemize}
\item Python is imperative and object oriented
\item Python interpreter interprets and executes Python code at runtime. (To be specific, though, a Python byte code .pyc could be ``compiled''from Python source code, though this pyc can be interpreted (CPython implementation), or just-in-time compiled. Python source can also be compiled to byte code for different platforms, like IronPython for .Net.)
\item Python type system uses duck typing, and is strongly, dynamically typed \cite{PythonStronglyTyped}
\item Standard Python implementation uses reference counting for garbage collection, with a supplemental GC facility to deal with reference cycles
\item Python syntax is designed to be highly readable, with a relatively uncluttered layout, uses English keywords frequently, and indentation to delimit program blocks \cite{ZenPython}
\item Python comes with good remote package management (de-facto management program, pip, now ships with official Python distribution)
\end{itemize}
The pros and cons of implementing DockAlt in Python include: 
\begin{itemize}
\item Pro: Good remote package management system. As of now ``Go get'' cannot specify a certain version of a package, Python pip can, and offers a variety of packages serving different purposes under different versions of Python in the meantime. These packages include LXC interfaces, which could be used for fast prototyping of DockAlt.
\item Pro: Easy to use, easy to read and contribute. Python is usually advertised for its simplicity. Syntax design of Python placed heavy emphasis on simplicity. This is a major reason that contributed to Python's success, and it'll likely attract a larger community, and making contributions from them easier to incorporate as well.
\item Con: Python interpreter could have rather bad performance. Interpretation is a relatively process, because the source code is not pre-compiled into an executable binary, but both analyzed and executed at runtime. Although many applications would sacrifice performance for code simplicity and readability, DockAlt may not be one of them, as when providing a lightweight platform on which various applications may be tried out, it's usually a plus if the platform implementation itself is as efficient as possible.
\item Con: Less reliable. Because Python is dynamically typed, type errors are caught at runtime, which may escape notice if the bugged code happens to be not covered in certain test runs. This is generally an unwanted feature for large software projects. However, this can be worked around with well designed software testing frameworks and test suites.
\end{itemize}

\section{DockAlt in Scala}

Scala advertises itself as a language with the advantages from object oriented and functional programming \cite{Scala}. Its features include
\begin{itemize}
\item Scala is objective oriented, every value is an object, and every operation is a method call.
\item Scala is functional. It has first-class functions, which means function are objects in Scala, and function type is just a regular class. Like most functional languages, Scala can do pattern matching over arbitrary classes, and has a general preference of immutability over mutation. Meanwhile, Scala allows gradual transition of paradigm: an imperative-language-like programming style is allowed.
\item Scala runs on JVM, Java and Scala classes can be easily mixed, and Java-based tools like ant, Eclipse, etc work seamlessly with Scala as well.
\item Scala is strongly and statically typed, and its type system implements parametric polymorphism and type inference. \cite{ScalaTypes}
\end{itemize}
Given these features, we summarize the pros and cons of using Scala as
\begin{itemize}
\item Pro: platform agnosticity. Since Scala byte code runs in JVM, it inherits the platform agnosticity advantage introduced by JVM.
\item Pro: safety and reliability. Similar with Java, a strongly and statically typed type system allows error checking at compile time, which is the preferred option over runtime typechecking for larger codebases.
\item Pro: inter-operability with Java. Java has a variety of useful built-in and third party packages. Since Scala can import from and export to Java classes, it would offer a wider range of tools to choose from, a rather mature native library to handle common needs, and the compiled package from Scala would be usable for the Java community as well.
\item Con: not so easy to use. Even though Scala allows an imperative programming style, its core is still functional; and functional programming paradigm should be applied to utilize the language's efficiency. These facts make the language harder to user. And given that the language has a smaller community than popular imperative languages like Python or Java, it's likely to attract less community contribution as well.
\end{itemize}

\section{References}

{\footnotesize \bibliographystyle{acm}
\bibliography{bibliography}}

\end{document}







